\documentclass[a4paper,openright,12pt]{article}
\usepackage[utf8]{inputenc}
\usepackage{graphicx} 
\usepackage{subfigure}
\usepackage[mathscr]{eucal}
\usepackage{titling}
\usepackage{float}
\usepackage{amsmath}
\usepackage{afterpage}
\usepackage{vmargin}
\usepackage[spanish,es-noshorthands]{babel}
\usepackage{csquotes}
\usepackage{eurosym} 
\usepackage{multirow}
\usepackage{graphicx}
\usepackage[export]{adjustbox}
\usepackage{url}
\usepackage[T1]{fontenc}
\usepackage{inconsolata}
\usepackage{amsfonts}
\usepackage[backend=biber, style=authoryear-icomp]{biblatex}
\usepackage[none]{hyphenat}
\sloppy
\usepackage[document]{ragged2e}
\usepackage{enumitem}
\usepackage{titlesec}
\setcounter{secnumdepth}{4}
\usepackage{datetime}
\usepackage{mfirstuc}
\setlist[enumerate]{itemsep=0mm}
\setlist[itemize]{itemsep=0mm}
\addbibresource{bibliography.bib}

% Packages for FSM
\usepackage{pgf}
\usepackage{tikz}
\usetikzlibrary{arrows,automata}

\usepackage{duckuments}

\setpapersize{A4}       %  DIN A4
\setmargins{3cm}        % margen izquierdo
{2cm}                   % margen superior
{15cm}                  % anchura del texto
{22.5cm}                % altura del texto
{10pt}                  % altura de los encabezados
{1cm}                   % espacio entre el texto y los encabezados
{0pt}                   % altura del pie de página
{2cm}                   % espacio entre el texto y el pie de página

\begin{document}

\author {Alonso Rodríguez}
\title {Trabajo Tutelado I}

% Título
\maketitle

% Justificamos el texto
\justifying

\section{Introducción}
En esta primera iteración de este trabajo tutelado, realizaremos una aproximación inicial al problema que como ingenieros
tratamos de resolver.

Nuestra misión es definir la máquina de estado finito (FSM) para un subconjunto del código morse, en concreto para los caracteres:
\begin{itemize}
    \item `A'
    \item `J'
    \item `S'
    \item `O'
    \item `2'
\end{itemize} 

\begin{center}
\begin{tikzpicture}[->,>=stealth, shorten >=1pt, auto, node distance=2.8cm, semithick]
  \tikzstyle{every state}=[,draw=black,text=black]

  \node[initial,state] (A)                    {$q_a$};
  \node[state]         (B) [above right of=A] {$q_b$};
  \node[state]         (D) [below right of=A] {$q_d$};
  \node[state]         (C) [below right of=B] {$q_c$};
  \node[state]         (E) [below of=D]       {$q_e$};

  \path (A) edge              node {0,1,L} (B)
            edge              node {1,1,R} (C)
        (B) edge [loop above] node {1,1,L} (B)
            edge              node {0,1,L} (C)
        (C) edge              node {0,1,L} (D)
            edge [bend left]  node {1,0,R} (E)
        (D) edge [loop below] node {1,1,R} (D)
            edge              node {0,1,R} (A)
        (E) edge [bend left]  node {1,0,R} (A);
\end{tikzpicture}
\end{center}


%% Pruebas con referencias
%\newpage
%Prueba: Como dice \textcite{milibro} Android mola, y estoy usando \LaTeX.
%Y además podemos hacer \parencite{misc-url}
\printbibliography[]{}
\end{document}
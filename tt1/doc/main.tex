\documentclass[a4paper,openright,12pt]{article}
\usepackage[utf8]{inputenc}
\usepackage{graphicx} 
\usepackage{subfigure}
\usepackage[mathscr]{eucal}
\usepackage{titling}
\usepackage{float}
\usepackage{amsmath}
\usepackage{afterpage}
\usepackage{vmargin}
\usepackage[spanish,es-noshorthands]{babel}
\usepackage{csquotes}
\usepackage{eurosym} 
\usepackage{multirow}
\usepackage{xcolor}
\usepackage{graphicx}
\usepackage[export]{adjustbox}
\usepackage{url}
\usepackage[T1]{fontenc}
\usepackage{inconsolata}
\usepackage{amsfonts}
\usepackage[backend=biber, style=authoryear-icomp]{biblatex}
\usepackage[none]{hyphenat}
\sloppy
\usepackage[document]{ragged2e}
\usepackage{enumitem}
\usepackage{titlesec}
\setcounter{secnumdepth}{4}
\usepackage{datetime}
\usepackage{mfirstuc}
\setlist[enumerate]{itemsep=0mm}
\setlist[itemize]{itemsep=0mm}
\addbibresource{bibliography.bib}

% Packages for FSM
\usepackage{pgf}
\usepackage{tikz}
\usetikzlibrary{arrows,automata}

% Things for Morse
\newcommand{\punto}{\kern+0.3pt\raisebox{0.35ex}{\huge\textbf.}}
\newcommand{\raya}{\kern+0.2pt\raisebox{-0.35ex}{\huge\textbf-}}

% Para generar imágenes mock
\usepackage{duckuments}

\setpapersize{A4}       %  DIN A4
\setmargins{3cm}        % margen izquierdo
{2cm}                   % margen superior
{15cm}                  % anchura del texto
{22.5cm}                % altura del texto
{10pt}                  % altura de los encabezados
{1cm}                   % espacio entre el texto y los encabezados
{0pt}                   % altura del pie de página
{2cm}                   % espacio entre el texto y el pie de página

\begin{document}

\author {Alonso Rodríguez}
\title {Trabajo Tutelado I}

% Título
\maketitle

% Justificamos el texto
\justifying{}

\section{Introducción}
En esta primera iteración de este trabajo tutelado, realizaremos una aproximación inicial al problema que como ingenieros
tratamos de resolver.

Nuestra misión es definir la máquina de estado finito (FSM) para un subconjunto del código morse, por simplificar la implementación.
En concreto debemos definirla, y posteriormente implementarla para los caracteres:
\begin{itemize}
    \item `A' (\punto\raya)
    \item `J' (\punto\raya\raya\raya)
    \item `S' (\punto\punto\punto)
    \item `O' (\raya\raya\raya)
    \item `2' (\punto\punto\raya\raya\raya)
\end{itemize} 

Además, se detalla cómo utilizar el LCD de la placa que vamos a utilizar en la segunda fase de este mismo trabajo para implementar este dispositivo morse,
la placa FRDM-KL46Z.

\section{Detalles acerca del Código Morse}
\begin{itemize}
    \item La unidad de tiempo no está definida, así que le llamaremos ``tick''
    \item Los puntos (\punto) duran 1 tick
    \item Las rayas (\raya) duran 3 ticks
    \item El espacio intra-carácter es de 1 tick
    \item El espacio inter-carácter es de 3 ticks
    \item El espacio inter-palabra es de 7 ticks
\end{itemize}


\section{Máquina de Estado Finito}
Para la implementación de esta FSM consideramos estados con tres posibles entradas:
\begin{itemize}
    \item Vacío ($\emptyset$): El usuario no pulsa el botón
    \item Punto (\punto): El usuario pulsa el botón durante 1 tick
    \item Raya (\raya): El usuario pulsa el botón durante 3 ticks
\end{itemize}

Cuando el usuario presiona el botón (SW1) para introducir un punto, el LED verde se enciende durante 1 tick, y cuando se pulsa el de la raya (SW2), durante 3 ticks.
Tras esto, se espera un tick, a no ser que estemos en un nodo terminal (en el que se escribe un carácter), en cuyo caso se espera tres ticks, y se enciende el led rojo.

Esto podría añadirlo a la FSM, pero considero que es algo sencillo de observar sin necesidad de saturar más la misma.

De todos modos, podemos extender la máquina de estados con este comportamiento, donde \punto \space significa pulsar SW1, y \raya \space SW2:

\begin{center}
\begin{tikzpicture}[->,>=stealth, shorten >=1pt, auto, node distance=3.2cm, semithick]
    \tikzstyle{every state}=[draw=black,text=black]

    % Nodo inicial
    \node[initial,state]        (vacio)                                     {$\frac{\emptyset}{\overset{\emptyset}{0}}$};

    % Puntos y rayas
    \node[state]                (punto)    [below right of=vacio]           {$\frac{\punto}{\overset{\punto}{\text{\textcolor{green}{\punto} + 1tick}}}$};
    \node[state]                (raya)     [above right of=vacio]           {$\frac{\raya}{\overset{\raya}{\text{\textcolor{green}{\punto} + 3tick}}}$};
    
    % Encendemos LED rojo y LCD
    \node[state, right=3cm, accepting]     (flush)    [right of=vacio]      {$\frac{\emptyset}{\overset{\text{\scriptsize{Muestra LCD}}}{\text{Enciende LED \textcolor{red}{\punto}}}}$};


    \path
        (vacio) edge                                node {\punto}           (punto)
                edge                                node {\raya}            (raya)
        (punto) edge [loop below]                   node {\punto}           (punto)
                edge [bend left=10, looseness=1]    node {\raya}            (raya)
                edge [bend right=10, looseness=1]   node {$\emptyset$}      (flush)
        (raya)  edge [bend left=10, looseness=1]    node {\punto}           (punto)
                edge [loop above]                   node {\raya}            (raya)
                edge [bend left=10, looseness=1]    node {$\emptyset$}      (flush)

    ;
\end{tikzpicture}
\end{center}

Para oxigenar el diagrama, en cada nodo que no se especifica el comportamiento de $\emptyset$, \punto, o \raya, significa que dada esa entrada, se vuelve al estado inicial.

\bigskip\bigskip\bigskip % Dejando espacio y eso

\begin{center}
\begin{tikzpicture}[->,>=stealth, shorten >=1pt, auto, node distance=3.2cm, semithick]
    \tikzstyle{every state}=[draw=black,text=black]

    % Nodo inicial
    \node[initial,state] (vacio)                              {$\frac{\emptyset}{\overset{\emptyset}{0}}$};

    % Camino a A, J
    \node[state]         (1p)     [below right of=vacio]      {$\frac{\punto}{\overset{\emptyset}{1}}$};
    \node[state]         (1p1r)   [below left  of=1p]         {$\frac{\punto\raya}{\overset{\emptyset}{1}}$};
    \node[state]         (1p2r)   [below left  of=1p1r]       {$\frac{\punto\raya\raya}{\overset{\emptyset}{1}}$};
    \node[state]         (1p3r)   [below left  of=1p2r]       {$\frac{\punto\raya\raya\raya}{\overset{\emptyset}{1}}$};

    % Camino a O
    \node[state]         (1r)     [below left  of=vacio]      {$\frac{\raya}{\overset{\emptyset}{1}}$};
    \node[state]         (2r)     [below left  of=1r]         {$\frac{\raya\raya}{\overset{\emptyset}{1}}$};
    \node[state]         (3r)     [below left  of=2r]         {$\frac{\raya\raya\raya}{\overset{\emptyset}{1}}$};

    % Camino a S, 2 desde 1p
    \node[state]         (2p)     [below right of=1p]         {$\frac{\punto\punto}{\overset{\emptyset}{1}}$};
    \node[state]         (3p)     [below right of=2p]         {$\frac{\punto\punto\punto}{\overset{\emptyset}{1}}$};
    \node[state]         (2p1r)   [below left  of=2p]         {$\frac{\punto\punto\raya}{\overset{\emptyset}{1}}$};
    \node[state]         (2p2r)   [below left  of=2p1r]       {$\frac{\punto\punto\raya\raya}{\overset{\emptyset}{1}}$};
    \node[state]         (2p3r)   [below left  of=2p2r]       {$\frac{\punto\punto\raya\raya\raya}{\overset{\emptyset}{1}}$};

    % Puntos finales
    \node[state]         (S)      [below of=3p]               {$\frac{\text{ `S' }}{\overset{\text{S}}{2}}$};
    \node[state]         (A)      [below left of=S]           {$\frac{\text{ `A' }}{\overset{\text{A}}{2}}$};
    \node[state]         (J)      [below of=1p3r]             {$\frac{\text{ `J' }}{\overset{\text{J}}{2}}$};
    \node[state]         (O)      [below of=3r]               {$\frac{\text{ `O' }}{\overset{\text{O}}{2}}$};
    \node[state]         (2)      [below of=2p3r]             {$\frac{\text{ `2' }}{\overset{\text{2}}{2}}$};

    \begin{pgfinterruptboundingbox}
    \path
        (vacio) edge                node {\punto}         (1p)
                edge                node {\raya}          (1r)
                edge [loop above]   node {$\emptyset$}    (vacio)
        (1p)    edge                node {\raya}          (1p1r)
                edge                node {\punto}         (2p)
        (1p1r)  edge                node {\raya}          (1p2r)
                edge [bend right=5] node {$\emptyset$}    (A)
        (1p2r)  edge                node {\raya}          (1p3r)
        (1p3r)  edge                node {$\emptyset$}    (J)
        (2p)    edge                node {\punto}         (3p)
                edge                node {\raya}          (2p1r)
        (3p)    edge                node {$\emptyset$}    (S)
        (2p1r)  edge                node {\raya}          (2p2r)
        (2p2r)  edge                node {\raya}          (2p3r)
        (2p3r)  edge                node {$\emptyset$}    (2)

        (1r)    edge                node {\raya}          (2r)
        (2r)    edge                node {\raya}          (3r)
        (3r)    edge                node {$\emptyset$}    (O)

        % Los de vuelta
        (A)    edge [bend left=240, looseness=1.5,in=250]          node {}                (vacio)
        (J)    edge [bend right=240,looseness=1.5,in=100]         node {}                (vacio)
        (S)    edge [bend left=240, looseness=1,  in=250]          node {}                (vacio)
        (O)    edge [bend right=240,looseness=1,  in=100]         node {}                (vacio)
        (2)    edge [bend right=240,looseness=1.75,  in=100]       node {}                (vacio)
    ;
    \end{pgfinterruptboundingbox}
\end{tikzpicture}
\end{center}

\bigskip\bigskip\bigskip

\section{Comportamiento del LCD}
Lorem ipsum


\printbibliography[]{}
\end{document}
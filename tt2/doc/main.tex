\documentclass[a4paper,openright,12pt]{article}

% Font config
\usepackage[utf8]{inputenc}
\usepackage{lmodern}
\usepackage[T1]{fontenc}
\usepackage{amsfonts}
\usepackage[babel=true]{microtype}
\usepackage{inconsolata}

% Figures, graphics, etc
\usepackage{graphicx}
\usepackage{caption}
\usepackage{subcaption}
\usepackage{float}

% Math symbols
\usepackage[mathscr]{eucal}
\usepackage{amsmath}

% Document Personalization and Language Configuration
\usepackage[spanish,es-noshorthands]{babel}
\usepackage{vmargin}
\usepackage{eurosym} 
\usepackage{xcolor}
\usepackage[export]{adjustbox}
\usepackage[document]{ragged2e}
\usepackage[shortlabels]{enumitem}

% Misc
\usepackage{titling}
\usepackage{titlesec}
\usepackage{afterpage}
\usepackage{csquotes}
\usepackage{multirow}
\usepackage{xurl}
\usepackage{hyperref}
\usepackage{listings}
\usepackage{array}
\usepackage{datetime}
\usepackage{mfirstuc}

% Special
\usepackage{minted} % Minted needs python installed, and pygments on path [https://pypi.org/project/Pygments/]

% Configurations
\setcounter{secnumdepth}{4}
\setlist[enumerate]{itemsep=0mm}
\setlist[itemize]{itemsep=0mm}

% No hyphenate
\usepackage[none]{hyphenat}
\sloppy

% Packages for FSM and diagrams
\usepackage{pgf}
\usepackage{tikz}
\usetikzlibrary{arrows,automata}

% Packages for generation mock images
\usepackage{duckuments}

% Things for Morse
\newcommand{\punto}{\kern+0.3pt\raisebox{0.35ex}{\huge\textbf.}}
\newcommand{\raya}{\kern+0.2pt\raisebox{-0.35ex}{\huge\textbf-}}
\usepackage[backend=biber, style=authoryear, backref=true, hyperref=true, urldate=long]{biblatex}
\addbibresource{bibliography.bib}

\setpapersize{A4}       %  DIN A4
\setmargins{3cm}        % margen izquierdo
{2cm}                   % margen superior
{15cm}                  % anchura del texto
{22.5cm}                % altura del texto
{10pt}                  % altura de los encabezados
{1cm}                   % espacio entre el texto y los encabezados
{0pt}                   % altura del pie de página
{2cm}                   % espacio entre el texto y el pie de página

\begin{document}

\author {Alonso Rodríguez}
\title {Trabajo Tutelado II}

% Título
\maketitle

% Justificamos el texto
\justifying{}


%%%%%%%%%%%%%%%%%%%%%%%%%%%%%%%%%%%%%%%%%%%%%%%%%%%%%%%%%%%%%%%%%%%%%%%%%%%%%%%%%%%%%%%%%%%%%%%%%%%%%%%%%%%%%%%%%%%%%%%%%
%                                                     INTRODUCCIÓN                                                      %
%%%%%%%%%%%%%%%%%%%%%%%%%%%%%%%%%%%%%%%%%%%%%%%%%%%%%%%%%%%%%%%%%%%%%%%%%%%%%%%%%%%%%%%%%%%%%%%%%%%%%%%%%%%%%%%%%%%%%%%%%
\section{Introducción}
El objetivo de esta práctica es programar un sistema que responda a los siguientes requerimientos:
\begin{itemize}    
    \item Se dan dos conjuntos de 10 ángulos, que se almacenan en dos arrays en grados, minutos y segundos.
    \item La tarea 1 determina cuales son complementarios y cuales suplementarios.
    \item La tarea 2 cuenta el número de ángulos complementarios, enciende el LED1 (verde - PTD5) y se muestra el resultado en el LCD, hasta que el usuario pulse SW1.
    \item La tarea 3 cuenta el número de ángulos suplementarios, muestra el resultado en el LCD y enciende el LED2 (Rojo - PTE29) hasta que el usuario pulse SW1.
    \item La tarea 4 cuenta el número de ángulos que no son ni complementarios ni suplementarios, muestra el resultado en el LCD y enciende alternativamente el LED1 y el LED2
          hasta que el usuario pulse SW1.
    \item Cuando las tareas han finalizados, la tarea 5 enciende ambos leds, hasta que se pulse SW1 o SW2.
\end{itemize}

El sistema se implementa sobre la placa FRDM-KL46Z.



%%%%%%%%%%%%%%%%%%%%%%%%%%%%%%%%%%%%%%%%%%%%%%%%%%%%%%%%%%%%%%%%%%%%%%%%%%%%%%%%%%%%%%%%%%%%%%%%%%%%%%%%%%%%%%%%%%%%%%%%%
%                                                    ESPECIFICACIÓN                                                     %
%%%%%%%%%%%%%%%%%%%%%%%%%%%%%%%%%%%%%%%%%%%%%%%%%%%%%%%%%%%%%%%%%%%%%%%%%%%%%%%%%%%%%%%%%%%%%%%%%%%%%%%%%%%%%%%%%%%%%%%%%
\clearpage
\section{Especificación}
\subsection{Interfaz con el Sistema}\label{design_button_translation}
La interacción con el sistema se realiza mediante los switches físicos SW1 y SW2. % TODO posible \ref (?)

\subsection{Flujo del Programa en Pseudocódigo}\label{pseudo_program_flow}
El programa debe inicializar las tareas al inicio de su ejecución (\ref{tasks}).

El funcionamiento de cada tarea será el siguiente:
\begin{itemize}
    \item init:
    \begin{samepage}    
    \begin{minted}{c}
// inicializar recursos
task5 = task_create()
// [...]
task1 = task_create()
task_delete_self()
    \end{minted}
    \end{samepage}
    \item task1:   
    \begin{minted}{c}
// Procesamos los elementos de los arrays
for(array...){
    result = compare_angles(array1, array2)

    if(result == complementary)
        // ángulos son complementarios
        // seteamos los parámetros apropiados
    else if(result == supplementary)
        // ángulos son suplementarios...
    else
        // ángulos no tienen ninguna relación...
    
    if(tarea_correspondiente_al_angulo_esta_ocupada)
        // si la tarea que maneja el tipo de ángulo detectado
        // está ocupada o pendiente de i/o con el usuario, tenemos
        // que esperar a que mande "ACK" conforme ya está lista
        os_event_wait_and(flag_del_angulo_correpondiente, NO_TIMEOUT)
    else
        // por esta rama solo se entra si la tarea no está esperando
        // input, es decir, la primera vez que encontramos un ángulo
        tarea_correspondiente_al_angulo_esta_ocupada = 1

    os_evt_set(flag_del_angulo_correspondiente, task_x);
}
// Al terminar el procesado, esperamos a que terminen todas las
// tareas para dar paso a la tarea 5, y matar el resto de tareas.
os_event_wait_and(tareas_que_quedan_por_terminar, NO_TIMEOUT)
os_task_delete(task2..4)
os_event_set(FLAG_TURNO_TASK5, task5)
task_delete_self()
    \end{minted}
    \item task2:
    \begin{samepage}    
    \begin{minted}{c}
while(1){
    // Esperamos a recibir un evento que nos indique que hay un
    // ángulo que debemos manejar
    result = os_event_wait_and(FLAG_ANGLE, NO_TIMEOUT)
    
    // Esperamos a poder tener acceso al LCD
    os_mutex_wait(lcd_mutex, NO_TIMEOUT)

    count++
    slcdDisplay(count)

    led_write(LED_GREEN, 1)
    while(!sw1){} //esperamos por input del usuario

    // Apagamos lo que encendiésemos
    led_write(LED_GREEN, 0)

    // Avisamos de que hemos terminado nuestra parte
    os_event_set(ACK_ANGLE_COMPLEMENTARY, task1)

    // Y liberamos el mutex cuando el usuario presiona el botón
    os_mutex_release(lcd_mutex)  
}
    \end{minted}
    \end{samepage}
    \item task3:
    \begin{samepage}    
    \begin{minted}{c}
while(1){
    // idem a task2 pero con LED_RED
}
    \end{minted}
    \end{samepage}
    \item task4:
    \begin{samepage}    
    \begin{minted}{c}
while(1){
    // idem a task2 pero con ambos led (alternativamente)
}
    \end{minted}
    \end{samepage}
    \item task5:
    \begin{samepage}    
    \begin{minted}{c}
// esperamos hasta que todas las tareas hayan terminado, momento
// en el cual nos llegará este evento
os_event_wait_and(FLAG_TURNO_TASK5, NO_TIMEOUT)

// mostramos nuestra salida
slcdClear();
led_write(LED_GREEN, 1)
led_write(LED_RED, 1)

while(!sw1 && !sw2){} //esperamos input

led_write(LED_GREEN, 0)
led_write(LED_RED, 0)

task_delete_self() // y terminamos
    \end{minted}
    \end{samepage}

\end{itemize}

\subsection{Diagrama de Tareas}\label{tasks}
\subsubsection{Tareas y eventos}
El diagrama muestra la comunicación que existe entre Task1 y el resto de las tasks, siguiendo el pseudocódigo propuesto en \ref{pseudo_program_flow}, en el que Task1 envía un evento
a la task correspondiente para que anote uno más y espere su turno para i/o con el usuario.

Además, se puede observar que hay dos tipos de líneas:
\begin{itemize}
    \item Las líneas sólidas indican una task activando eventos de otra.
    \item Las líneas rayadas indican los ``ACK'' que envían cada una de las tareas para indicar que están preparadas para recibir más datos.
    \item La línea rayada de init a Task1 no sabía muy bien como representarla, pero básicamente está ahí para que init pueda hacer acto de presencia, e indicar que debido a que
          el resto de tareas esperan, a efectos prácticos, al morir init, la única task que no espera eventos es Task1, por lo que le ``cede el control''.
\end{itemize}

Por legibilidad no se ha incluido la creación de las tareas, ya que todas son creadas por init al inicio del programa.

\subsubsection{Representación}

\bigskip

\begin{center}
\begin{tikzpicture}[->,>=stealth, shorten >=1pt, auto, node distance=3.2cm, semithick]
    \tikzstyle{every state}=[draw=black,text=black]

    % Nodo inicial
    \node[initial,state] (init)                                 {$\text{init}$};

    % Tareas
    \node[state, right=0.5cm]           (task1)   [below right of=init]        {$\text{Task1}$};
    \node[state, right=0.8cm]           (task2)   [above right of=task1]       {$\frac{\text{Task2}}{\text{\textcolor{green}{\punto}}}$};
    \node[state, right=0.8cm]           (task3)   [right       of=task1]       {$\frac{\text{Task3}}{\text{\textcolor{red}{\punto}}}$};
    \node[state, right=0.8cm]           (task4)   [below right of=task1]       {$\frac{\text{Task4}}{\text{\textcolor{green}{\punto} $\leftrightarrow$ \textcolor{red}{\punto}}}$};
    \node[state, right=2cm]             (task5)   [right       of=task4]       {$\frac{\text{Task5}}{\text{\textcolor{green}{\punto} \& \textcolor{red}{\punto}}}$};

    \begin{pgfinterruptboundingbox}
    \path[sloped,anchor=south]
        (task1)  edge [bend left=6,looseness=0.3]      node {COMPL}     (task2)
                 edge [bend left=6,looseness=0.3]      node {SUPL}      (task3)
                 edge [bend left=6,looseness=0.3]      node {OTRO}      (task4)
    ;
    \path[sloped,anchor=north]
        (task1)  edge [bend right=75,looseness=0.75]   node {FINAL}          (task5)
    ;
    \path[dashed]
        (task2)  edge [bend left=6,looseness=0.3]      node {}          (task1)
        (task3)  edge [bend left=6,looseness=0.3]      node {}          (task1)
        (task4)  edge [bend left=6,looseness=0.3]      node {}          (task1)
        (init)  edge []                                node {}          (task1)
    ;
    \end{pgfinterruptboundingbox}
\end{tikzpicture}
\end{center}


%%%%%%%%%%%%%%%%%%%%%%%%%%%%%%%%%%%%%%%%%%%%%%%%%%%%%%%%%%%%%%%%%%%%%%%%%%%%%%%%%%%%%%%%%%%%%%%%%%%%%%%%%%%%%%%%%%%%%%%%%
%                                                    IMPLEMENTACIÓN                                                     %
%%%%%%%%%%%%%%%%%%%%%%%%%%%%%%%%%%%%%%%%%%%%%%%%%%%%%%%%%%%%%%%%%%%%%%%%%%%%%%%%%%%%%%%%%%%%%%%%%%%%%%%%%%%%%%%%%%%%%%%%%
\clearpage
\section{Implementación}
\subsection{Decisiones de diseño}
\subsubsection{Diseño de la concurrencia}\label{concurrencia_angulos}
El enunciado de la práctica da lugar a múltiples interpretaciones, por lo que he tenido que replantear la práctica un par de veces. Tal cual
lo entiendo ahora mismo, se entiende que el flujo del programa es el siguiente:
\begin{enumerate}
    \item Se inicia la placa y la tarea 1 calcula el primer ángulo, por lo que se mostrará un 1 en el LCD y se encenderá(n) el LED correspondiente.
    \item Al presionar SW1 se pasa a la siguiente tarea, es decir, si el siguiente ángulo era del mismo tipo se mostrará un 2 en el LCD, manteniendo el LED estático.
          Por otro lado si el siguiente ángulo fuese diferente, se mostrará un 1 y cambiará el patrón de los LEDs.
    \item Al terminar de presionar SW1 tantas veces como ángulos queremos comparar, se apaga el LCD y se encienden los dos LEDs: Hemos entrado en la tarea 5.
    \item Al pulsar SW1 o SW2 se apaga todo y no se hace nada más. 
\end{enumerate}

Es decir, que tomamos una aproximación ciertamente limitada por el número de inputs que puede realizar el usuario. Si el usuario no pulsa SW1, no seguimos contando.

Esto último me ha dado ciertos dolores de cabeza, ya que inicialmente la tarea 1 debía contar en segundo plano, y con SW1 se iba cambiando secuencialmente entre las tareas 2, 3 y 4,
viendo el estado del contador interno de cada una de ellas ``en directo''. Tras comentar con compañeros qué entendían en el enunciado, y tras la revisión, parece que hay un consenso
en que se debe hacer de la manera que he implementado, y \emph{espero} que esté bien.

\bigskip

Se ha optado por comunicación a través de eventos, ya que si bien los mensajes se han tenido en consideración, creo que los eventos facilitan la concurrencia en este escenario, y es que
tras comentar en la revisión que ``se debía aprovechar concurrencia'' estuve dandole vueltas a que, realmente, si únicamente hacemos comunicación a través de eventos, y no computamos
el siguiente ángulo hasta que no pulsamos el switch, no deja de ser un programa puramente secuencial, que hemos partido en tasks por hacernos la vida más difícil.

\bigskip

Debido a esto se ha implementado un sistema que aprovecha la concurrencia de una forma interesante: El siguiente ángulo siempre se computa \emph{ahead of time} (\ref{concurrencia_angulos}),
siendo el \emph{speedup} de este cambio mínimo en este caso, pero haciéndose mucho más notable si el cálculo a realizar tuviese un tiempo de procesado más elevando.

De esta forma, podemos computar hasta tres ángulos \emph{ahead of time}, \emph{e.g.}
\begin{enumerate}
    \item Ángulos complementarios -> Se activa evento en task2 -> Bloquea el mutex
    \item Ángulos suplementarios -> Se activa evento en task3 -> Espera por el mutex
    \item Ángulos sin relación -> Se activa evento en task4 -> Espera por el mutex
    \item Ángulos complementarios -> Espera a que task2 mande el ack
    \item Usuario pulsa SW1 -> task2 libera el mutex y manda ack a task1 -> task3 bloquea el mutex -> task2 recibe el ángulo complementario anteriormente en espera y bloquea el mutex de nuevo
    \item \ldots
\end{enumerate}

\bigskip

O en forma de tabla:

\begin{center}
\begin{tabular}{ | >{\centering\arraybackslash}m{3cm} | >{\centering\arraybackslash}m{3cm} | >{\centering\arraybackslash}m{3cm} | >{\centering\arraybackslash}m{3cm} | >{\centering\arraybackslash}m{0.8cm} |}
    \hline
    Task1                   &   Task2        & Task3 & Task4 & i/o\\
    \hline
    Complementarios         &   Recibe evento, bloquea mutex & - & - & \textcolor{green}{\text{\punto}} \\
    \hline
    Suplementarios          &   - & Recibe evento, espera mutex & - & \textcolor{green}{\text{\punto}} \\
    \hline
    Sin Relación            &  - & - & Recibe evento, espera mutex & \textcolor{green}{\text{\punto}} \\
    \hline
    Complementarios, espera a que task2 mande ack para enviar evento         &  - & - & - & \textcolor{green}{\punto} SW1\\
    \hline
    Recibe ack de task2, y envia evento a task2   &   Libera mutex y ack a task1, espera mutex & Bloquea mutex & - & \textcolor{red}{\text{\punto}}\\
    \hline
\end{tabular}
\end{center}

\bigskip

En el peor de los casos, siempre tendremos un ángulo en espera, \emph{e.g. solamente medimos ángulos complementarios}:
\begin{enumerate}
    \item Ángulos complementarios -> Se activa evento en task2 -> Bloquea el mutex
    \item Ángulos complementarios -> Espera a que task2 mande el ack
    \item Usuario pulsa SW1 -> task2 libera el mutex y manda ack a task1 -> task2 recibe el ángulo complementario anteriormente en espera y bloquea el mutex de nuevo
    \item Ángulos complementarios -> Espera a que task2 mande el ack
    \item \ldots
\end{enumerate}

\begin{center}
\begin{tabular}{ | >{\centering\arraybackslash}m{3cm} | >{\centering\arraybackslash}m{3cm} | >{\centering\arraybackslash}m{3cm} | >{\centering\arraybackslash}m{3cm} | >{\centering\arraybackslash}m{0.8cm} |}
    \hline
    Task1                   &   Task2        & Task3 & Task4 & i/o\\
    \hline
    Complementarios         &   Recibe evento, bloquea mutex & - & - & \textcolor{green}{\text{\punto}} \\
    \hline
    Complementarios, espera a que task2 mande ack para enviar evento         &  - & - & - & \textcolor{green}{\punto} SW1\\
    \hline
    Recibe ack de task2, y envia evento a task2   &   Libera mutex y ack a task1, bloquea mutex & - & - & \textcolor{green}{\text{\punto}}\\
    \hline
\end{tabular}
\end{center}

\subsubsection{Método de lectura}
Al igual que en el anterior trabajo tutelado, se ha optado por entrada salida por interrupciones, ya que requiere menos CPU, y en general, es lo suficientemente sencilla como
para permitirnos reutilizarla para esta práctica.

\subsubsection{Corner cases}\label{corner_cases}
No se han identificado corner cases, ya que cada tarea deja a lo largo de su ciclo todos los dispositivos de i/o ``como los encontró''.

Todas las tareas parten de que los LEDs están apagados, y los switches SW1 y SW2 despulsados. Para esto todas las tareas al terminar su ejecución (\ref{generic_counting_tasks})
ejecutan la función \texttt{clear\_phy()}:
\begin{samepage}
\begin{minted}{c}
    void clear_phy(void){
        setLedG(0); setLedR(0);
        sw1 = 0; sw2 = 0;
    }
\end{minted}
\end{samepage}

\subsubsection{Inicialización de los Dispositivos de i/o}
Los dispositivos de entrada (switches) y salida (LEDs) no se pueden inicializar cuando el sistema RTX está iniciado, por lo que se han de inicializar en la función \texttt{main} antes
de iniciar el sistema operativo:
\begin{minted}{c}
    int main (void) {
        // Inicializar rutinas de interrupción
        initSwitch1(sw1_pressed);
        initSwitch3(sw2_pressed);
        
        // Inicializar sLcd
        slcdInitialize();
        
        // Inicializar los LEDs poniendolos a apagado
        setLedG(0);
        setLedR(0);
        
        os_sys_init(init);
    }
\end{minted}



\subsection{Detalles de Implementación}
Creo que este trabajo no necesita de tante explicación como el anterior, donde había que explicar cómo se interpretaban los estados, etc.

La aproximación ha sido la siguiente, la tarea 1 tenemos que programarla \emph{ad-hoc} si o si, ya que hace una función única. Sin embargo, las tareas 2 a 4 realizan la misma función
exceptuando que interaccionan con unos eventos diferentes, y encienden los LEDs con patrones diferentes, por lo que se han creado las ``tareas genéricas de conteo'', denotadas por
\texttt{task\_generic}, que como se comenta en \ref{generic_counting_tasks}, a parte de simplificar el código y seguir el principio DRY, reduce el tamaño total del programa, haciendolo
más cómodo de leer y mantenible, debido a que no hay que propagar los cambios a tres versiones casi idénticas de las tareas que cuentan ángulos.

La tarea 5 también se ha diseñado e implementado \emph{ad-hoc}.

\subsubsection{Tarea 1}
Esta tarea recorre los dos arrays de ángulos, de tamaño fijo, denotado por \texttt{ANGLE\_ARRAY\_LEN}:
\begin{minted}{c}
    #define ANGLE_ARRAY_LEN 10
\end{minted}

Al recorrerlo tenemos tres posibles resultados: 
\begin{itemize}
    \item \texttt{ANGLE\_COMPL}: Ángulo Complementario
    \item \texttt{ANGLE\_SUPPL}: Ángulo Suplementario
    \item \texttt{ANGLE\_NANGL}: Ángulos sin Relación (\texttt{\textcolor{red}{NANGL} = \textcolor{red}{N}onrelated \textcolor{red}{ANGL}es})
\end{itemize}

\bigskip

Sabiendo la relación entre los ángulos, podemos clasificarlos con el siguiente switch, el cual asigna el evento y tid que se han de emplear para comunicarse con la tarea correspondiente:
\begin{minted}{c}
    switch(result){
        case ANGLE_COMPL:
            curr_flag = FLAG_ANGLE_COMPL;
            curr_tid = t_2;
            break;
        case ANGLE_SUPPL:
            curr_flag = FLAG_ANGLE_SUPPL;
            curr_tid = t_3;
            break;
        case ANGLE_NANGL:
            curr_flag = FLAG_ANGLE_NANGL;
            curr_tid = t_4;
            break;
    }
\end{minted}

Y este bloque de código se encarga de enviar los eventos, y realizar la espera cuando una tarea ya está ocupada:
\begin{minted}{c}
    if(tasks_waiting & curr_flag){
        os_evt_wait_and(curr_flag, 0xFFFF);
    }else{
        tasks_waiting |= curr_flag;
    }

    os_evt_set(curr_flag, curr_tid);
\end{minted} 

\bigskip

Finalmente cuando todas las tareas han terminado, se matan directamente desde la tarea 1. Esta es una decisión que debido al tamaño del programa creo que es completamente lógica: No hay
tareas colgando de las tareas que cuentan, y crear un evento similar al \texttt{SIGTERM} de Linux sería complicar el código innecesariamente cuando tenemos una función que hace muy bien 
lo que promete y es mucho más sencilla.
\begin{minted}{c}
    os_evt_wait_and(tasks_waiting, 0xFFFF);
    os_tsk_delete(t_2);
    os_tsk_delete(t_3);
    os_tsk_delete(t_4);
    os_evt_set(1, t_5);
    
    os_tsk_delete_self(); // Cuando hayamos terminado, salimos
\end{minted}

Línea importante aquí es \texttt{os\_evt\_wait\_and(tasks\_waiting, 0xFFFF);}, que hace que si por ejemplo nuestro programa solamente ha contado ángulos complementarios no espere por
las tareas de los suplementarios y los sin clasificar, ya que nunca enviarán nada, y quedará bloqueado.

\subsubsection{Tareas Genéricas de Conteo}\label{generic_counting_tasks}
Tras una idea feliz antes de ir a dormir el día anterior a cuando pensaba entregar, se me ocurrió la \emph{ahora obvia} idea de que podía juntar las tareas de conteo en una sola tarea
genérica, no para ganar en velocidad, pero sí para ganar en mantenibilidad y ser más rigurosos con el principio DRY, a parte de reducir el tamaño del código, puesto que tenemos menos código
de repetido para las funciones de conteo, a cambio de una ligeramente mayor y genérica.

\bigskip

Así, y siguendo el pseudocódigo en \ref{pseudo_program_flow}, la implementación de la tarea genérica es:
\begin{minted}{c}
    __task void task_generic(void* argv){
        uint16_t count = 0;
        
        #define my_flag (*(uint16_t*) argv)
        //uint16_t my_flag = *(uint16_t*) argv;
        
        while(1){
            if(os_evt_wait_or(my_flag, 0xFFFF) == OS_R_EVT){
                
                os_mut_wait(mut_1, 0xFFFF);
                
                count++;
                slcdDisplay(count, 10);
                
                switch(my_flag){
                    case FLAG_ANGLE_COMPL:
                        setLedG(1);
                        while(!sw1){os_dly_wait(DEFAULT_WAIT_TIME);}
                        break;
                    case FLAG_ANGLE_SUPPL:
                        setLedR(1);
                        while(!sw1){os_dly_wait(DEFAULT_WAIT_TIME);}
                        break;
                    case FLAG_ANGLE_NANGL:
                        while(!sw1){
                            setLedG(0);
                            setLedR(1);
                            os_dly_wait(DEFAULT_WAIT_TIME);
                            setLedG(1);
                            setLedR(0);
                            os_dly_wait(DEFAULT_WAIT_TIME);
                        }
                        break;
                }

                clear_phy();
                os_evt_set(my_flag, t_1);
                
                os_mut_release(mut_1);
            }
        }
        #undef my_flag
    }
\end{minted}

En esta tarea podemos observar varias cosas, la primera y más evidente es la línea:
\begin{minted}{c}
    #define my_flag (*(uint16_t*) argv)
    //uint16_t my_flag = *(uint16_t*) argv;
\end{minted}
que sustituye a la asignación de la variable que había anteriormente en su lugar, y que por claridad se ha dejado debajo como comentario. De esta forma se evita una instrucción de
asignación, el uso de memoria en el stack, y no hay inconvenientes, puesto que el tiempo de acceso es el mismo. Lo que si, es necesario hacer \texttt{\#undef my\_flag} al final de la
función para imitar el funcionamiento de cuando la variable cae \emph{out of scope}, es decir, poder utilizar otras variables llamadas \texttt{my\_flag} a lo largo del código. 

\bigskip

Continuando tenemos el switch de \texttt{my\_flag}, que define el comportamiento particular de la función, según escuche un tipo u otro de ángulos:
\begin{minted}{c}
    switch(my_flag){
        case FLAG_ANGLE_COMPL:
            setLedG(1);
            while(!sw1){os_dly_wait(DEFAULT_WAIT_TIME);}
            break;
        case FLAG_ANGLE_SUPPL:
            setLedR(1);
            while(!sw1){os_dly_wait(DEFAULT_WAIT_TIME);}
            break;
        case FLAG_ANGLE_NANGL:
            while(!sw1){
                setLedG(0);
                setLedR(1);
                os_dly_wait(DEFAULT_WAIT_TIME);
                setLedG(1);
                setLedR(0);
                os_dly_wait(DEFAULT_WAIT_TIME);
            }
            break;
    }
\end{minted}

Al terminar, como se menciona en \ref{corner_cases}, se ejecuta \texttt{clear\_phy()}, para dejar la i/o como se encontró.


%%%%%%%%%%%%%%%%%%%%%%%%%%%%%%%%%%%%%%%%%%%%%%%%%%%%%%%%%%%%%%%%%%%%%%%%%%%%%%%%%%%%%%%%%%%%%%%%%%%%%%%%%%%%%%%%%%%%%%%%%
%                                                  DESCRIPCIÓN EVENTOS                                                  %
%%%%%%%%%%%%%%%%%%%%%%%%%%%%%%%%%%%%%%%%%%%%%%%%%%%%%%%%%%%%%%%%%%%%%%%%%%%%%%%%%%%%%%%%%%%%%%%%%%%%%%%%%%%%%%%%%%%%%%%%%
\clearpage
\section{Intercomunicación mediante Eventos}
\subsection{Motivación}
Los sistemas informáticos necesitan comunicación entre ellos, y para satisfacer esta necesidad existen múltiples paradigmas.
En nuestro caso necesitamos comunicación entre procesos, y como se explica en el temario de la asignatura, existen múltiples
opciones para realizar esto mismo, entre otras:
\begin{itemize}
    \item Comunicación mediante Eventos
    \item Comunicación mediante Mensajes (Uso de una Mailbox)
\end{itemize}

\subsection{Decisión de Diseño}
En este caso se ha escogido la comunicación por eventos, a pesar de haber pensado inicialmente en una comunicación mediante mensajes, debido a que considero
que puedo aprender más utilizando los eventos (sistema que no he usado como tal), y queda un sistema definitivamente dinámico, en el que prácticamente usamos los elementos como
``paquetes enrutados'', que llegan a receptores en una red, y además siendo utilizados también como método de señalizado a modo de testigo.

Todo esto desde mi punto de vista hace que resulte un sistema conceptualmente bastante satisfactorio, en el que podemos visualizar una intensiva comunicación inter-proceso.

\subsection{Explicación detallada}
No he encontrado tanta información como me gustaría acerca de este tema (\textcolor{blue}{solicitar sugerencia de bibliografía}), a parte de en la documentación de \emph{ARM
Keil} \autocite[\text{Event Flag Management}]{keil_function_reference}.

Por el momento podemos comentar que contamos con 6 rutinas, con las que podemos realizar toda la gestión de eventos:
\begin{samepage}
\begin{center}
\begin{tabular}{ | c | >{\centering\arraybackslash}m{11cm} | }
    \hline
    Rutina              &   Descipción\\
    \hline
    \texttt{os\_evt\_clr}        &   Elimina uno o más event flags de una tarea\\
    \hline
    \texttt{os\_evt\_get}        &   Devuelve las eventa flags que hicieron salir del \texttt{os\_evt\_wait\_or}\\
    \hline
    \texttt{os\_evt\_set}        &   Activa una o más flags de un evento de una tarea\\
    \hline
    \texttt{os\_evt\_wait\_and}  &   Espera a que se activen todos los flags necesarios\\
    \hline
    \texttt{os\_evt\_wait\_or}   &   Espera a que se active alguno de los flags necesarios\\
    \hline
    \texttt{isr\_evt\_set}       &   Activa una o más flags de un evento de una tarea, sólo activable desde dentro de una rutina de interrupción\\
    \hline
\end{tabular}
\end{center}
\end{samepage}

\subsection{Comportamiento Apreciado}
Aquí serán completadas todos los matices que apreciemos con este sistema en el que aún no he programado, por lo que es muy probable que se me
escapen cosas actualmente que al momento de la implementación se añadirán aquí y donde sea correspondiente.


\clearpage
\begin{flushleft}
\printbibliography[]{}
\end{flushleft}
\end{document}